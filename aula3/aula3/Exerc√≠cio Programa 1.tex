\documentclass{article}
\title{EP 1 - Somador de bytes}
\date{}
\usepackage[utf8]{inputenc}

\begin{document}
	\maketitle
	Neste exercício vamos implementar um somador de números binários limitados a 1 $byte$.\\
	Você deve criar uma função: $somaByte(a, b)$ que recebe 2 $bytes$
	$a$ e $b$ e devolve $c$, tal que $c = a + b$.\\ Veja que um $byte$ tem limite de 8 bits 
	(explicado abaixo) e você deve tratar o overflow da forma que preferir, uma vez que
	sua escolha será parte do exercício.\\
	\center
	O que é 1 $byte$:\\
		1 $byte$ é uma unidade de informação digital equivalente a oito $bits$.\\
		Para passar o número 6, por exemplo, para um $byte$ fazemos:
		 $$(2^7\times 0) + (2^6\times 0) + (2^5\times 0) + (2^4\times 0) + (2^3\times 0) + (2^2\times 1) + (2^1\times 1) + (2^0\times 0)$$
		 Pois $2^2 + 2^1 = 4 + 2 = 6$ e ficamos com o $byte$ $000000110$\\
		 \hfill
		\newline
		\hfill
		\newline		 
		\hfill
		\newline
		 A soma de um byte é calculada da seguinte forma:
		\begin{table}[h]
			\center
			\begin{tabular}{lll}
				- & - - - - -1- - &   \\
				2 & 00000 010       & + \\ 
				3 & 00000 011       &   \\ \cline{1-2}
				5 & 00000 101       &  
			\end{tabular}
		\end{table}
		\newline
		Veja o que acontece se somarmos:
		\begin{table}[h]
			\center
			\begin{tabular}{lll}
				- & 1  - - - - - - - - &   \\
				 128 &  \ \ 10000000       & + \\ 
				128 &   \  \ 10000000       &   \\ \cline{1-2}
				256 &   \ \ 00000000       &  
			\end{tabular}
		\end{table}
		\newline
		Precisaríamos de mais 1 $bit$ para armazenar o valor, porém o $byte$ é limitado a 8 $bits$, dizemos, portanto,
		que ocorreu $overflow$, ou seja, estouramos o limite da representação.\\
		Você deve tratar esse fenômeno fazendo com que seu programa não ``quebre'' e que devolva algo significativo
		neste caso.\\
		\center
			Iremos discutir suas soluções! Bom Trabalho e mãos à obra!
		
		
		
	
\end{document}
