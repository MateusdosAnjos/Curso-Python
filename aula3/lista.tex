\documentclass{article}
\usepackage[portuguese]{babel}
\usepackage[utf8]{inputenc}
\title{Lista de exercícios para aula 3}	
\date{28/03/2019}
\begin{document}
	\maketitle
	Escreva um programa que:
	\begin{enumerate}
		\item
			Recebe um número $n$ e
			decide se $n$ é maior que zero.
		\item
			Recebe $n$ valores e
			calcula a média destes $n$ valores.
		\item
			Recebe um número $n$ e decide se $n$ é primo
		\item
			Ordena $n$ elementos:
			\subitem
				Utilizando o método .sort() em uma lista.
			\subitem			
				Criando sua própria função de ordenação.
		\item
			Receba um número inteiro $n$ e decida se $n$ é triangular.
			(Um número natural é triangular se ele é produto de três 	
			números naturais consecutivos. Exemplo: 120 é triangular, 
			pois 4.5.6 = 120.)	
		\item
			Resolva o problema do Professor (está no PDF sobre
			listas)
		\item
			Dado uma sequência de $n$ números inteiros, determine 	
			a soma dos números pares. 	
		\item
			Dado um número inteiro $n$ determine todos os triângulos
			pitagóricos cujo perímetro seja igual a $n$.\\
			(Um triângulo é pitagórico se: 
			$hipotenusa^2 = (cateto_1)^2 + (cateto_2)^2$)
		\item
			Dado um número $n$ e um conjunto de listas 
			$L = {l_1, l_2, l_3, \dots, l_n}$ devolva a posição
			de $n$ em cada uma das listas.
			(Vocês devem definir a melhor forma de lidar com o caso
			de $n$ não aparecer na lista)
		\item
			Leia um arquivo .txt e devolva a frequência das palavras
			no texto.\\
			(Adicional: Habilitar opção Ordem Alfabética ou Ordem de
			Frequência)	
	\end{enumerate}
	Divirtam-se :D
\end{document}\grid
